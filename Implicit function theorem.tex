\documentclass{article}
\usepackage{amsmath}
\usepackage{amssymb}
\usepackage{upgreek}
\usepackage{latexsym}
\usepackage{arydshln}%这两个是矩阵虚线用的
\usepackage[thmmarks,amsmath]{ntheorem}
\theoremstyle{nonumberplain}
\theoremheaderfont{\bfseries}
\theorembodyfont{\normalfont}
\theoremsymbol{$\square$}
\newtheorem{Proof}{\hskip 2em Proof}
\begin{document}
\textbf{Implicit function theorem.} 
\emph{Let \textbf{$\Upphi$} be of class C$^{(q)}$ from an open set D$\subset$E$^n$ into E$^m$, where q$\geq$1 and 1$\leq$m\emph{<}n. Let \textbf{x}$_0$$\in$D be such that 
\textbf{$\Upphi$}\emph{(}\textbf{x}$_0$\emph{)}
\emph{=} \emph{\textbf{0}}
and 
$\tilde{J}$\textbf{$\Upphi$}\emph{(}\textbf{x}$_0$\emph{)}
\emph{$\neq$ 0}.
 Then there exists a neighbourhood U of \textbf{x}$_0$, an open set R$\subset$E$^r$
  containing $\widehat{\textbf{x}}$$_0$, 
  and }\textbf{$\upphi$} 
  \emph{\emph{= (}$\phi$$^1$,$\dotsc$,$\phi$$^m$\emph{)} of class C$^{(q)}$ on R such that:}
$$\tilde{J}\textbf{$\Upphi$}(\textbf{x}) \neq 0  \qquad
\emph{for all}  
\quad\textbf{x}\in\emph{U}\quad; $$
\emph{and}

$$\{\textbf{x}\in\emph{U}:\textbf{$\Upphi$}(\textbf{x}) = \textbf{0}\} = \{(\hat{\textbf{x}},\textbf{$\upphi$}(\hat{\textbf{x}})):\hat{\textbf{x}}\in R\}$$
\begin{Proof}
Since $\Upphi$ is at least of C$^{(1)}$, the Jacobian $\tilde{J}$\textbf{$\Upphi$}  
is a continuous function. By assumption it is not zero at \textbf{x}$_0$, and therefore is not zero for \textbf{x} in some neighbourhood \emph{U}$_0$ of \textbf{x}$_0$. 

Let us consider the transformation \textbf{f}, with domain \emph{U}$_o$ and values in \emph{E$^n$},
$$f^i(\textbf{x}) = x^i,\qquad\qquad i= 1,\dotsc,r,$$
$$f^{r+1}(\textbf{x}) = \Upphi^l(\textbf{x}),\qquad l = 1,\dotsc,m,\quad m + r = n.$$

The transformation \textbf{f} is of class C$^{(q)}$. Its matrix of partial derivatives is
$$\begin{pmatrix}
\begin{array}{cccc:ccc}
1&0&\dotsc&0&0&\dotsc&0\\
0&1&\dotsc& &\dotsc&\dotsc&0\\
\vdots& &\ddots& & & &\vdots\\ 
0&0&\dotsc&1&0&\dotsc&0\\ \hdashline
\Upphi^1_1& &\dotsc&\Upphi^1_r&\Upphi^1_{r+1}& &\Upphi^1_n\\
\vdots& & & & & &\vdots\\
\Upphi^m_1& &\dotsc&\Upphi^m_r&\Upphi^m_{r+1} &\ddots&\Upphi^m_n\\
\end{array}
\end{pmatrix}$$
By properties of determines, the Jacobian \emph{J}\textbf{f}(\textbf{x}) equals the determinant \emph{$\tilde{J}$}\textbf{$\Upphi$}(\textbf{x}) of the m $\times$ m block in the lower right-hand corner. Therefore \emph{J}\textbf{f}(\textbf{x}) $\neq$ 0. By the inverse function theorem, there is a neighbourhood \emph{U} of \textbf{x}$_0$ such that \textbf{f}(\emph{U}) is an open set and the restriction \textbf{f}|\emph{U} has an inverse \textbf{g} of class C$^{(q)}$.\par
Writing(\textbf{$\hat{x}$},\textbf{0}) for (x$^1$,$\dotsc$,x$^r$,0,$\dotsc$,0), let
$$\emph{R} = \{\hat{\textbf{x}}:
(
\hat{\textbf{x}},
\textbf{0})\in\textbf{f}(\emph{U}   )\}.$$
Since \textbf{f}(\emph{U}) is an open set, \emph{R} is open. For every $\hat{x} \in$ \emph{R},let
$$ \phi^l(\hat{\textbf{x}}) = \emph{g}^{r+l}(\hat{\textbf{x}},\textbf{0})
,\qquad \emph{l} = 1,\dotsc,m. 
$$
Then \textbf{x}$\in$\emph{U} and \textbf{$\Upphi$}(\textbf{x}) = \textbf{0} if and only if $\hat{\textbf{x}} \in$ \emph{R} and \textbf{f}(\textbf{x}) = ($\hat{\textbf{x}}$,\textbf{0}). Since \textbf{f}|\emph{U} and \textbf{g} are inverses, \textbf{f}(\textbf{x}) = ($\hat{\textbf{x}}$,\textbf{0}) if and only if \textbf{x} = \textbf{g} = ($\hat{\textbf{x}}$,\textbf{0})
\end{Proof}
\end{document}